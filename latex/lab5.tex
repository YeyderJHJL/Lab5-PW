\documentclass[12]{article}
\usepackage[spanish]{babel}
\usepackage{graphicx}
\usepackage{wrapfig}
%tabla
\usepackage{booktabs}
\usepackage{colortbl}
\usepackage{adjustbox}
%color
\usepackage[table]{xcolor}
\usepackage{xcolor}
\definecolor{red}{RGB}{255,0,0}
\definecolor{blackcolor}{rgb}{0.95,0.95,0.92}
\definecolor{codegreen}{rgb}{0,0.6,0}
\definecolor{codegray}{rgb}{0.5,0.5,0.5}
\definecolor{codepurple}{rgb}{0.58,0,0.82}

%para código
\usepackage{listings}

\lstdefinestyle{mystyle}{
	backgroundcolor=\color{blackcolor},
	commentstyle=\color{codegreen},
	keywordstyle=\color{magenta},
	numberstyle=\tiny\color{codegray},
	stringstyle=\color{codepurple},
	basicstyle=\ttfamily\footnotesize,
	breakatwhitespace=false,
	breaklines=true,
	captionpos=t, % b
	keepspaces=true,
	numbers=left,
	showspaces=false,
	showstringspaces=false,
	showtabs=false,
	tabsize=2,
}
\lstset{extendedchars=true, inputencoding=utf8}

\usepackage[colorlinks=true, urlcolor=blue, linkcolor=black]{hyperref}%referencias de 
\usepackage{fancyvrb}
\renewcommand{\familydefault}{\sfdefault}%cambio de familya de texto
\graphicspath{{./images/}}%dirección de las imágenes
\usepackage[left=25mm, right=25mm, top=35mm, bottom=30mm, headheight=35mm]{geometry}
\usepackage{lastpage}% devuelve la última página
%encabezado
\usepackage{fancyhdr}
\pagestyle{fancy}
%conf encabazado
\lhead{\includegraphics[width=0.20\textwidth]{sistemas}}
\chead{\textbf{{\footnotesize UNIVERSIDAD NACIONAL DE SAN AGUSTIN\\
			FACULTAD DE INGENIERÍA DE PRODUCCIÓN Y SERVICIOS\\
			DEPARTAMENTO ACADÉMICO DE INGENIERÍA DE \\ SISTEMAS E INFORMÁTICA\\
			ESCUELA PROFESIONAL DE INGENIERÍA DE SISTEMAS}}}
\rhead{\includegraphics[width=0.1\textwidth]{abet}}
%conf pie de página
\lfoot{Estudiante \autor}
\cfoot{Pweb2}
\rfoot{Pag. \thepage\ - \pageref{LastPage}}

%comandos adicionales
\newcommand{\autor}{Jhamil Yeyder Turpo Añasco}
\renewcommand{\headrulewidth}{1.5pt}
\renewcommand{\footrulewidth}{1pt}
%\graphicspath{{./images/}} indica la posición de las figuras
%opening



\begin{document}
	
	\begin{center}
		\textbf{GUÍA DE LABORATORIO 5}
	\end{center}
	
	\begin{table}[h]
		\centering
		
		\begin{tabular}{|p{0.2\linewidth}|p{0.12\linewidth}|p{0.15\linewidth}|p{0.12\linewidth}|p{0.16\linewidth}|p{0.09\linewidth}|}
			\hline
			%\cellcolor{red} solo una casilla
			\rowcolor{red}
			\multicolumn{6}{|c|}{\textbf{INFORMACIÓN BÁSICA}} \\
			\hline
			ASIGNATURA: & \multicolumn{5}{l|}{Programación Web 2} \\
			\hline
			TEMA: & \multicolumn{5}{l|}{Django} \\
			\hline
			NÚMERO DE PÁCTICA: & 05 & AÑO LECTIVO: & 2023 A & NRO. SEMESTRE: & III \\
			\hline
			FECHA DE INICIO: & 29-May-2023 & FACHA FIN: & 05-Junio-2023 & DURACIÓN: & 4 horas \\
			\hline
			\multicolumn{6}{|l|}{AUTOR:} \\
			\multicolumn{6}{|l|}{\autor} \\
			\hline
			\multicolumn{6}{|l|}{DOCENTE:} \\
			\multicolumn{6}{|l|}{Anibal Sardon Paniagua} \\
			\hline
		\end{tabular}
		\label{tab: datos}
	\end{table}
	
	
	\section*{\begin{center}
			\textbf{{\Huge DJANGO}}
	\end{center}}
	
	\section{COMPETENCIAS DEL CURSO}
	
	\begin{itemize}
		\item Diseña responsablemente aplicaciones web, sus componentes o procesos para satisfacer necesidades dentro de restricciones realistas: económicas, medio ambientales, sociales, políticas, éticas, de salud, de seguridad, manufacturación y sostenibilidad.
		
		\item Construye responsablemente soluciones con tecnología web siguiendo un proceso adecuado llevando a cabo las pruebas ajustada a los recursos disponibles del cliente.
		
		\item Aplica de forma flexible técnicas, métodos, principios, normas, esténdares y herramientas del desarrollo web necesarias para la construcción de aplicaciones web e implementación de estos sistemas en una organización.
	\end{itemize}
	
	\section{EJERCICIOS PROPUESTOS}
	
	\lstlistoflistings
	\begin{itemize}
		\item Cree la aplicación Library paso a paso desde la siguiente url:
		\item \url{https://developer.mozilla.org/en-US/docs/Learn/Server-side/Django/Tutorial_local_library_website}
		\begin{Verbatim}[xleftmargin=\parindent]
		\end{Verbatim}
		%\begin{figure}[h!]
			%\centering
			%\includegraphics[width=0.2\linewidth%]{url}
			%\caption{ejercicio 2a}
			%\label{fig:2a}
		%\end{figure}
	\end{itemize}
	\newpage
	\begin{lstlisting}[caption={Estructura de los repositorios y archivos}, style=mystyle]
	Library
	|   db.sqlite3
	|   manage.py
	|
	+---Apps
	|   \---catalog
	|       |   admin.py
	|       |   apps.py
	|       |   models.py
	|       |   tests.py
	|       |   urls.py
	|       |   views.py
	|       |   __init__.py
	|       |
	|       +---migrations
	|       +---static
	|       |   \---css
	|       |           styles.css
	|       |
	|       +---Templates
	|       |   |   base.html
	|       |   |   index.html
	|       |   |
	|       |   +---catalog
	|       |   |       author_detail.html
	|       |   |       author_list.html
	|       |   |       book_detail.html
	|       |   |       book_list.html
	|       |   |
	|       |   \---registration
	|       |           logged_out.html
	|       |           login.html
	|       |           password_reset_complete.html
	|       |           password_reset_confirm.html
	|       |           password_reset_done.html
	|       |           password_reset_email.html
	|       |           password_reset_form.html
	|       |
	|       \---__pycache__
	|
	+---images
	|       django1.png
	|       django2(esqueleto).png
	|       django3(admin).png
	|       django4(administracion).png
	|       django5(viewpersonal).png
	|       django6(libros).png
	|       django7(paginacion).png
	|       django8(autores).png
	|       django9(autor).png
	|
	\---Library
	|   asgi.py
	|   settings.py
	|   urls.py
	|   wsgi.py
	|   __init__.py
	|
	\---__pycache__
	\end{lstlisting}
	\newpage
	Los archivos más importantes que se verán a continuación son:
	\begin{itemize}
		%\lstinputlisting[language=HTML, style=mystyle, linerange={10-20, 30-40}]{tu_archivo.html}
		%\lstinputlisting[language=HTML, style=mystyle, firstline=10, lastline=20]{tu_archivo.html}
		
		\item En Library:
		\lstinputlisting[language=Python, caption={Ajustes que se toman en cuenta en settings}, style=mystyle, linerange={12-13,33-41, 55-69, 107-109, 126-132}]{../Library/Library/settings.py}
		
		\lstinputlisting[language=Python, caption={Código de urls del proyecto}, style=mystyle, linerange={17-27, 32-36}]{../Library/Library/urls.py}
		\newpage
		\item En la App catalog:
		
		\lstinputlisting[language=Python, caption={Código de models (campos creados)}, style=mystyle]{../Library/Apps/catalog/models.py}
		
		\lstinputlisting[language=Python, caption={Código de urls de la aplicación catalog (redireccionamientos)}, style=mystyle]{../Library/Apps/catalog/urls.py}
		
		\lstinputlisting[language=Python, caption={Código de vistas (administración de contenido)}, style=mystyle]{../Library/Apps/catalog/views.py}
	\end{itemize}
	
	
	\section{TAREA}
	\begin{itemize}
		\item Elabore un primer informe grupal de la aplicación que desarrollará durante este semestre. 
		\item repositorio grupal: \\
		\url{https://github.com/AndreRH09/Proyecto_Pweb2.git}
		\item Utilicen todas las recomendaciones dadas en la aplicación library.
	\end{itemize}
	
	\section{PREGUNTAS}
	
	\begin{itemize}
		\item Por cada integrante del equipo, resalte un aprendizaje que adquirió al momento de estudiar Django. No se reprima de ser detallista. Coloque su nombre entre parentesis para saber que es su aporte.\\
		
		En lo personal he adquirido amplio conocimiento sobre el desarrollo web con Django, abarcando aspectos como el enrutamiento, modelos y bases de datos, plantillas, formularios y seguridad. Esto les ha permitido desarrollar aplicaciones web robustas y escalables utilizando el framework Django.
		También Aprendí a utilizar el sistema de enrutamiento de Django para definir las URL de mi aplicación y asignar las vistas correspondientes. También comprendí la importancia de mantener una estructura de archivos organizada para facilitar el desarrollo y la localización de los recursos.
	\end{itemize}\newpage
	
	\subsection*{Referencias}
	
	\begin{itemize}
		\item \url{https://developer.mozilla.org/en-US/docs/Learn/Server-side/Django/Tutorial_local_library_website}
		\item \url{https://github.com/mdn/django-locallibrary-tutorial}
		\item \url{https://github.com/rescobedoq/pw2/tree/main/labs/lab05}
	\end{itemize}
	
\end{document}

